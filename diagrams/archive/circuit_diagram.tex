\documentclass[10pt,oneside]{article}


\usepackage{graphics}
\usepackage{rotating}
\usepackage{subfigure}
\usepackage{circuitikz}


\begin{document}
\title{Circuit diagrams for a Buccaneer procedures trainer}
\author{Steven~Hancock}

\maketitle

\pagebreak
\tableofcontents
\listoffigures
\pagebreak

\section{Introduction}
This contains the circuit diagrams needed for a Buccaneer procedures trainer, driven by an Arduino using a real Buccaneer cockpit.


\section{Diagrams}
This section contains circuit diagrams for each instrument, and then a master diagram showing all the wiring.

\subsection{Engine RPM gauge}
The engine RPM gauge is a 3-phase delta type motor. It runs at up to approximately 5,100 rpm (85 Hz) and can achieve full speed with a 12 V DC supply. Each coil has a resistance of 40 \(\Omega\). The input wires are marked ``A'', ``B'' and ``C''.


\begin{figure}[!hbtp]
\centering
\begin{circuitikz} \draw
(0,0) to[battery] (0,4)
  to[ammeter] (4,4) -- (4,0)
  to[lamp] (0,0)
;
\end{circuitikz}
\caption{\small{Circuit diagram test}}
\label{FIGtest}
\end{figure}


\subsection{Jet pipe temperature gauge (TGT)}
The JPT gauge is a simple voltage reader. An input of 0 V sets the needle at min deflection and an input of approximately 18 mV sets the needle at max deflection. The instrument resistance is 33.5 \(\Omega\). The circuit diagram is shown in figure~\ref{FIGjpt_driver}.

\begin{figure}[!hbtp]
\centering
\begin{circuitikz} \draw
(0,0) [battery,l=PWM]{} to (0,2) -- (0,2)
  to[R=9.3 \(k \Omega\)] (0,4) -- (0,6)
  to[voltmeter, l=18 mV] (4,6) -- (4,4)
  to [C=22 \(\mu f\)] (0,4) -- (4,4)
  to (4,0) node[ground]{}
;
\end{circuitikz}
\caption{\small{JPT driver circuit diagram. The voltmeter represents the JPT gauge and the voltage input the Arduino PWM output pin.}}
\label{FIGjpt_driver}
\end{figure}





\section*{Acknowledgements}
Thanks to Scott Bouchard for providing the circuit description needed to drive the RPM gauges. Thanks to Peter Newman and Adam Tanton for help with the practical electronics.



\end{document}


